\chapter{Motivation}

\section{Die historische Entwicklung JavaScripts}

JavaScript hat als Lingua Franca des World Wide Webs und primäre clientseitige Programmiersprache für Webanwendungen aller Art innerhalb der letzten Jahre enorm an Bedeutung gewonnen. Dies belegt beispielsweise die alljährliche Umfrage \enquote{\emph{Stack Overflow Developer Survey}} der Programmierer-Plattform \emph{Stack Overflow}, welche die Ergebnisse der Befragung von über 100.000 Software-Entwicklern weltweit auswertet~\autocite{stackoverflow:survey:2018}. Bereits das sechste Jahr in Folge führt JavaScript die Rangliste der populärsten Programmiersprachen an. Dies ist nicht verwunderlich hinsichtlich des ungebrochenen Trends, dass immer mehr Software als Webanwendung konzipiert wird~\autocite{taivalsaari:2017}\autocite{casteleyn:2014:ria}. Viele beliebte Applikationen wie \emph{Spotify}, \emph{Slack} oder \emph{Visual Studio Code} basieren z.~B. auf dem Framework \emph{Electron}~\autocite{electron}. Dabei wird die gesamte grafische Oberfläche der Anwendung durch HTML und CSS innerhalb eines Chromium-Webbrowsers realisiert. Derzeitige JavaScript-Frameworks und -Bibliotheken wie Angular~\autocite{angular} oder React~\autocite{react} verdeutlichen den hohen Bedarf an modernen Entwicklungsansätzen und anspruchsvollen Programmierparadigmen, die es ermöglichen, umfangreiche, skalierbare Anwendungen in JavaScript umzusetzen.

Als JavaScript jedoch 1995 als Bestandteils des Browsers \emph{Netscape Communicator} erfunden worden ist\footnote{Zunächst \enquote{LiveScript} genannt.}~\autocite{severance:2012:js10days}, war nicht abzusehen, welche große Bedeutung die Sprache über 20 Jahre später inne haben wird. Ursprünglich war die Skriptsprache lediglich als ergänzendes Werkzeug gedacht, um den Zugriff auf das \emph{Document Object Model} (DOM) von Websites zu ermöglichen und diese dynamischer zu gestalten.
% FIXME CITATION NEEDED
Die zu Anfang sehr inkonsistente Implementierung der Sprache in den verschiedenen Webbrowsern und die schwache Typisierung JavaScripts war (und ist) Grund für große Frustration und anfängliche Ablehnung der Sprache seitens professioneller Software-Entwickler~\autocite{oreilly:2001:js}.
Mit Aufkommen des Web 2.0~\autocite{oreilly:2005:web20} und des damit einhergehenden sukzessiven Bedeutungszuwachses von JavaScript\footnote{Beispielsweise als Bestandteil von \emph{AJAX} (Asynchronous JavaScript and XML)~\autocite{garret:ajax}.} begann die Sprache jedoch allmählich zu reifen. Die sechste Version der als \emph{ECMA Script} bekannten Sprachspezifikation \enquote{ECMA Script 2015}\footnote{Oft auch als \enquote{ES6} bezeichnet.}~\autocite{ecmascript:2015} versuchte viele der historisch bedingten Schwachstellen auszumerzen und führte eine Vielzahl von syntaktischen Verbesserungen sowie neuen Funktionen ein.

Hierdurch und durch die stetige Weiterentwicklung von ECMA Script wurde der Grundstein für zukunftssichere JavaScript-Anwendungen gelegt. Dennoch stellt insbesondere das dynamische, schwache Typsystems der Programmiersprache ein Hindernis für die Entwicklung umfangreicher Software dar. Unbeabsichtigte, implizite Typumwandlungen zur Laufzeit und falsche Annahmen über vorliegende Datenstrukturen sind häufige Fehlerursachen. Eine starke, statische Typisierung, wie sie beispielsweise in C++ oder Haskell vorliegt ist erstrebenswert, denn sie bietet viele Vorteile für den Software-Entwicklungsprozess: Logik- und Flüchtigkeitsfehler im Quelltext können oftmals bereits vor Ausführung des Programms erkannt und behoben werden. Die Entwickler gewinnen Sicherheit und Zuversicht, dass Änderungen in umfangreichen Projekten keine unerwünschte Nebenwirkung verursachen, wodurch sich die Wartbarkeit der Software erhöht. Eine explizite Typisierung führt darüber hinaus dazu, dass der Programmierer gezwungen ist seine \emph{Intension} klar zu formulieren. Hierdurch verbessert sich die Ausdruckskraft des Codes. Weiterhin wird der Quelltext an Ort und Stelle durch eine vernünftige Typisierung bereits grundlegend dokumentiert (\enquote{\emph{inline documentation}}).

