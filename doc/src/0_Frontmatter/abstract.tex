\clearpage
\pdfbookmark[section]{Zusammenfassung}{abstract}

\begin{abstract}
Statische Typsysteme unterstützen die Entwicklung korrekter, sicherer und wartbarer Software. Die vorliegende Masterarbeit beschäftigt sich mit der Transpilierung zweier Systeme, die eine statische Typisierung in JavaScript ermöglichen. Dabei wird einerseits \textit{Flow}~\autocite{FLOW:PAPER}, andererseits \textit{TypeScript}~\autocite{TYPESCRIPT:SPEC} betrachtet. In dieser Arbeit wird die Implementierung eines Transcompilers auf Basis von \textit{Babel}~\autocite{BABEL} ausgeführt, der beliebige durch Flow typisierte JavaScript-Programme in äquivalenten TypeScript-Code übersetzen kann. Mittels des umgesetzten Transpilers wurden zwei reale Projekte des Unternehmens \textit{Spreadshirt} erfolgreich nach TypeScript migriert. Es konnte daraufhin auf Grundlage empirischer Daten, die anhand der übersetzten Projekte gewonnen wurden, gezeigt werden, dass TypeScript Flow in einigen Aspekten überlegen ist: So wurden durch den Wechsel des Typsystems neue Programmfehler aufgedeckt und es konnte belegt werden, dass TypeScript eine umfangreichere Unterstützung für externe Software-Bibliotheken mittels Typdefinitionen bietet. Darüber hinaus wird TypeScript deutlich transparenter weiterentwickelt als Flow. Schließlich konnte in einigen Fällen bei TypeScript ein besseres Laufzeitverhalten der Typüberprüfungen gegenüber Flow nachgewiesen werden.
\end{abstract}
