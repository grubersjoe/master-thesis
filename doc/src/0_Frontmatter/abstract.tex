\clearpage
\pdfbookmark[section]{Zusammenfassung}{abstract}

\begin{abstract}
Die vorliegende Masterarbeit behandelt die Systeme Flow~\autocite{FLOW:PAPER} und TypeScript~\autocite{TYPESCRIPT:SPEC}, die eine statische Typisierung in JavaScript ermöglichen. Dabei wird die Implementierung eines Transpilers auf Basis von Babel~\autocite{BABEL} ausgeführt, der beliebige durch Flow typisierte JavaScript-Programme in äquivalenten TypeScript"=Code übersetzen kann. Mittels des entwickelten Transpilers wurden zwei reale Projekte des Unternehmens Spreadshirt erfolgreich nach TypeScript migriert. Es kann auf Grundlage empirischer Daten, die anhand der übersetzten Projekte gewonnen wurden, gezeigt werden, dass TypeScript verschiedene Vorteile im Vergleich zu Flow aufweist: So sind durch den Wechsel des Typsystems neue Programmfehler aufgedeckt worden, und es kann belegt werden, dass TypeScript eine umfangreichere Unterstützung für externe Software-Bibliotheken mittels Typdefinitionen bietet. Darüber hinaus geht aus der Untersuchung hervor, dass TypeScript transparenter als Flow entwickelt wird. Schließlich kann in einigen Fällen bei TypeScript ein besseres Laufzeitverhalten der Typüberprüfungen gegenüber Flow nachgewiesen werden.
\end{abstract}
