\begingroup
\setlength{\tabcolsep}{6pt} % reset to default
\begin{table}[tb]
  \caption{Vergleich verschiedener Werkzeuge zur Transpilierung von JavaScript"=Quelltexten.}
  \footnotesize
  \begin{tabu} to \textwidth {@{}lllccccccrrX@{}}
    \midrule
    \libertineSB{Werkzeug} & \libertineSB{Typ} & \libertineSB{Format} & \libertineSB{Erw.} & \libertineSB{ES10} & \libertineSB{ES10+} & \libertineSB{Flow} & \libertineSB{TS} & \libertineSB{JSX} & \libertineSB{Aktivität} & \libertineSB{Sterne} \\
    \midrule
    Acorn     & P  &  ESTree  & \pie{2} & \pie{2} & \pie{1} & \pie{0} & \pie{0} & \pie{2} &   5 &  5.250 & {\autocite{ACORN}} \\ % 2012
    Astring   & G  &  ESTree  & \pie{2} & \pie{2} & \pie{1} & \pie{0} & \pie{0} & \pie{0} &  61 &    500 & {\autocite{ASTRING}} \\ % 2015
    Babel     & PG &  Babel   & \pie{2} & \pie{2} & \pie{2} & \pie{2} & \pie{2} & \pie{2} & 225 & 34.650 & {\autocite{BABEL}} \\ % 2014
    Escodegen & G  &  ESTree  & \pie{0} & \pie{0} & \pie{0} & \pie{0} & \pie{0} & \pie{0} &   2 &  1.900 & {\autocite{ESCODEGEN}} \\ % 2012
    Esprima   & P  &  ESTree  & \pie{0} & \pie{1} & \pie{0} & \pie{0} & \pie{0} & \pie{1} &  12 &  5.450 & {\autocite{ESPRIMA}} \\ % 2011
    Recast    & PG &  diverse & \pie{2} & \pie{2} & \pie{2} & \pie{2} & \pie{2} & \pie{2} &  43 &  2.700 & {\autocite{RECAST}} \\ % 2012
    \midrule
  \end{tabu}
  \vspace{0.75\baselineskip}
  \caption*{
    \footnotesize
    \begin{minipage}[t]{.4\linewidth}
      {
        \renewcommand{\arraystretch}{1.1}
        \begin{tabular}{@{}ll@{}}
          P & Parser\\
          G & Codegenerator\\
          \pie{0} & keine Unterstützung\\
          \pie{1} & teilweise Unterstützung\\
          \pie{2} & vollständige Unterstützung\\
        \end{tabular}
      }
    \end{minipage}
    \begin{minipage}[t]{.55\linewidth}
      {
        \renewcommand{\arraystretch}{1.1}
        \begin{tabular}{@{}ll@{}}
          Erw. & Erweiterbarkeit\\
          ES10 & ECMAScript 2019\\
          ES10+ & vorgeschlagene JavaScript-Erweiterungen\\
          TS & TypeScript\\
          Sterne & Anzahl der Sterne auf GitHub\\
        \end{tabular}
      }
    \end{minipage}

    \vspace{\baselineskip}
    Stand: Oktober 2019
    \begin{justify}
      Aktivität: Entspricht der Summe der Zahl von \emph{Merge Requests}, neuer bzw. geschlossener Fehlerberichte, veröffentlichter Git-Commits auf dem Hauptzweig und der Anzahl beteiligter Autoren innerhalb eines Monats auf der Plattform GitHub~\autocite{GITHUB}.
    \end{justify}
  }
  \label{tab:transpilers}
\end{table}
\endgroup
