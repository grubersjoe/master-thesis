\chapter{Grundlagen}

... und diese werden nun sogleich näher beschrieben:

\section{Statische Typsysteme für JavaScript}

  Es gibt noch viele weitere: https://github.com/jashkenas/coffeescript/wiki/List-of-languages-that-compile-to-JS

  \subsection{Flow}
    Flow beschreiben (und zwar mit entsprechender Fachsprache)

    \subsubsection{Basistypen}
    \subsubsection{Hilfstypen}
    \subsubsection{Deklarationen}

  \subsection{TypeScript}
    TS beschreiben (und zwar mit entsprechender Fachsprache)

\section{Compiler und Transpiler}

  Was macht eigentlich so ein Compiler bzw. Transpiler? Hier Theorie (AST etc.)


  \subsection{Lexikalische Analyse}

    Quelltext (string) => Tokens

    Parser, Tokenizer = Lexer

  \subsection{Syntaxanalyse}

    Tokens => AST


  \subsection{Parser und Transpiler für JavaScript}
  \label{subsec:js-transpilers}

    Babel als populärer Vertreter eines JavaScript-Compilers
