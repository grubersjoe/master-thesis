\chapter{Grundlagen}

... und diese werden nun sogleich näher beschrieben:

\section{Statische Typsysteme für JavaScript}

Es gibt noch viele weitere: https://github.com/jashkenas/coffeescript/wiki/List-of-languages-that-compile-to-JS

\subsection{Flow}
  Flow beschreiben (und zwar mit entsprechender Fachsprache)

\subsubsection{Basistypen}

\begin{table}
  \begin{tabular}{@{}ll@{}}
    \toprule
    \textbf{Basis-Typ}               & \textbf{Beispiel}                                           \\
    \midrule
    Array type                 & \texttt{Array<{}number>{}}                               \\
    Boolean literal type       & \texttt{true}                                            \\
    Boolean type               & \texttt{boolean}                                         \\
    Empty type                 & \texttt{empty}                                           \\
    Exact object type          & \texttt{\{| prop: any |\}}                               \\
    Function type              & \texttt{(string, \{\}) => number}                        \\
    Generic type annotation    & \texttt{let v: <{}FlowType>{}}                           \\
    Generics                   & \texttt{type Generic<{}T: Super> = T}                    \\
    Interface type             & \texttt{interface \{ +prop: number \}}                   \\
    Intersection type          & \texttt{type Intersection = T1 \& T2}                    \\
    Mixed type                 & \texttt{mixed}                                           \\
    Null literal type          & \texttt{null}                                            \\
    Nullable type (Maybe type) & \texttt{?number}                                         \\
    Number literal type        & \texttt{42}                                              \\
    Number type                & \texttt{number}                                          \\
    Object type                & \texttt{\{ {[}string{]}: number \}}                      \\
    Opaque type                & \texttt{opaque type Opaque = number}                     \\
    String literal type        & \texttt{'literal'}                                       \\
    String type                & \texttt{string}                                          \\
    This type                  & \texttt{this}                                            \\
    Tuple type                 & \texttt{{[}Date, number{]}}                              \\
    Type alias                 & \texttt{type Type = <{}FlowType>{}}                      \\
    Type casting               & \texttt{(variable: string)}                              \\
    Typeof type                & \texttt{typeof undefined}                                \\
    Union type                 & \texttt{number | null}                                   \\
    Void type                  & \texttt{void}                                            \\
    \bottomrule
  \end{tabular}
  \caption{Basistypen von Flow~\autocite{FLOW_TYPE_ANNOTATIONS} mit Beispiel}
  \label{tab:flow-base-types}
\end{table}

\subsubsection{Hilfstypen}

\begin{table}
  \begin{tabular}{@{}ll@{}}
    \toprule
    \textbf{Flow-Typ}   & \textbf{Beispiel}              \\
    \midrule
    Call                & \texttt{\$Call<F, T...>}       \\
    Class               & \texttt{Class<T>}              \\
    Difference          & \texttt{\$Diff<A, B>}          \\
    Element type        & \texttt{\$ElementType<T, K>}   \\
    Exact               & \texttt{\$Exact<T>}            \\
    Existential type    & \texttt{*}                     \\
    Keys                & \texttt{\$Keys<T>}             \\
    None maybe type     & \texttt{\$NonMaybeType<T>}     \\
    Object map          & \texttt{\$ObjMap<T, F>}        \\
    Object map with key & \texttt{\$ObjMapi<T, F>}       \\
    Property type       & \texttt{\$PropertyType<T, k>}  \\
    ReadOnly            & \texttt{\$ReadOnly<T>}         \\
    Rest                & \texttt{\$Rest<A, B>}          \\
    Shape               & \texttt{\$Shape<T>}            \\
    Tuple map           & \texttt{\$TupleMap<T, F>}      \\
    Values              & \texttt{\$Values<T>}           \\
    Subtype             & \textit{deprecated}            \\
    Supertype           & \textit{deprecated}            \\
    \bottomrule
  \end{tabular}
  \caption{Flows Hilfstypen~\autocite{FLOW_UTILITY_TYPES} mit Beispiel}
  \label{tab:flow-utility-types}
\end{table}

\subsubsection{Deklarationen}

\subsubsection{Typ-Importe und -Exporte}

\begin{table}
  \begin{tabular}{@{}ll@{}}
    \toprule
    \textbf{Typ}               & \textbf{Beispiel}                      \\
    \midrule
    \texttt{Type imports}     & \texttt{import type T from './types'}   \\
    \texttt{Type exports}     & \texttt{export type T = number | null}  \\
  \end{tabular}
  \caption{Weitere Sprachkonstrukte von Flow}
  \label{tab:flow-other-constructs}
\end{table}


\subsection{TypeScript}
  TS beschreiben (und zwar mit entsprechender Fachsprache)

\section{Compiler und Transpiler}

  Was macht eigentlich so ein Compiler bzw. Transpiler? Hier Theorie (AST etc.)


  \subsection{Lexikalische Analyse}

    Quelltext (string) => Tokens

    Parser, Tokenizer = Lexer

  \subsection{Syntaxanalyse}

    Tokens => AST


  \subsection{Parser und Transpiler für JavaScript}
  \label{subsec:js-transpilers}

    Babel als populärer Vertreter eines JavaScript-Compilers
