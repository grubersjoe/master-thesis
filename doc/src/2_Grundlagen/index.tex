\chapter{Grundlagen}

... und diese werden nun sogleich näher beschrieben:

\section{Statische Typsysteme für JavaScript}

Es gibt noch viele weitere: https://github.com/jashkenas/coffeescript/wiki/List-of-languages-that-compile-to-JS

\subsection{Flow}
  Flow beschreiben (und zwar mit entsprechender Fachsprache)

\subsubsection{Basistypen}

\begin{table}
  \begin{tabular}{@{}ll@{}}
    \toprule
    \textbf{Flow-Typ}               & \textbf{Beispiel}                                                 \\
    \midrule
    \texttt{Array type}                 & \texttt{Array<{}number>{}}                               \\
    \texttt{Boolean literal type}       & \texttt{true}                                            \\
    \texttt{Boolean type}               & \texttt{boolean}                                         \\
    \texttt{Empty type}                 & \texttt{empty}                                           \\
    \texttt{Exact object type}          & \texttt{\{| prop: any |\}}                               \\
    \texttt{Function type}              & \texttt{(string, \{\}) => number}                        \\
    \texttt{Generic type annotation}    & \texttt{let v: <{}FlowType>{}}                           \\
    \texttt{Generics}                   & \texttt{type Generic<{}T: Super> = T}                    \\
    \texttt{Interface type}             & \texttt{interface \{ +prop: number \}}                   \\
    \texttt{Intersection type}          & \texttt{type Intersection = T1 \& T2}                    \\
    \texttt{Mixed type}                 & \texttt{mixed}                                           \\
    \texttt{Null literal type}          & \texttt{null}                                            \\
    \texttt{Nullable type (Maybe type)} & \texttt{?number}                                         \\
    \texttt{Number literal type}        & \texttt{42}                                              \\
    \texttt{Number type}                & \texttt{number}                                          \\
    \texttt{Object type}                & \texttt{\{ {[}string{]}: number \}}                      \\
    \texttt{Opaque type}                & \texttt{opaque type Opaque = number}                     \\
    \texttt{String literal type}        & \texttt{'literal'}                                       \\
    \texttt{String type}                & \texttt{string}                                          \\
    \texttt{This type}                  & \texttt{this}                                            \\
    \texttt{Tuple type}                 & \texttt{{[}Date, number{]}}                              \\
    \texttt{Type alias}                 & \texttt{type Type = <{}FlowType>{}}                      \\
    \texttt{Type casting}               & \texttt{(variable: string)}                              \\
    \texttt{Typeof type}                & \texttt{typeof undefined}                                \\
    \texttt{Union type}                 & \texttt{number | null}                                   \\
    \texttt{Void type}                  & \texttt{void}                                            \\
    \bottomrule
  \end{tabular}
  \caption{Basistypen von Flow mit Beispiel}
  \label{tab:flow-base-types}
\end{table}

\subsubsection{Hilfstypen}


\subsubsection{Deklarationen}

\subsubsection{Typ-Importe und -Exporte}

\begin{table}
  \begin{tabular}{@{}ll@{}}
    \toprule
    \textbf{Typ}               & \textbf{Beispiel}                      \\
    \midrule
    \texttt{Type imports}     & \texttt{import type T from './types'}   \\
    \texttt{Type exports}     & \texttt{export type T = number | null}  \\
  \end{tabular}
  \caption{Syntax von Typexporten und -importen}
  \label{tab:flow-base-types}
\end{table}


\subsection{TypeScript}
  TS beschreiben (und zwar mit entsprechender Fachsprache)

\section{Compiler und Transpiler}

  Was macht eigentlich so ein Compiler bzw. Transpiler? Hier Theorie (AST etc.)


  \subsection{Lexikalische Analyse}

    Quelltext (string) => Tokens

    Parser, Tokenizer = Lexer

  \subsection{Syntaxanalyse}

    Tokens => AST


  \subsection{Parser und Transpiler für JavaScript}
  \label{subsec:js-transpilers}

    Babel als populärer Vertreter eines JavaScript-Compilers
