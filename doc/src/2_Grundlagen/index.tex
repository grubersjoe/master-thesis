\chapter{Grundlagen}

... und diese werden nun sogleich näher beschrieben:

\section{Statische Typsysteme für JavaScript}

  Es gibt noch viele weitere: https://github.com/jashkenas/coffeescript/wiki/List-of-languages-that-compile-to-JS

  \subsection{Flow}
    Flow beschreiben (und zwar mit entsprechender Fachsprache)

    \subsubsection{Basistypen}
    \subsubsection{Hilfstypen}
    \subsubsection{Deklarationen}

  \subsection{TypeScript}
    TS beschreiben (und zwar mit entsprechender Fachsprache)

\section{Quelltext-Transformation durch Transpilierung}

  Was macht eigentlich so ein Transpiler? Hier Theorie (AST etc.)

  \subsection{Theoretische Grundlagen}

    Parser, AST etc.


  \subsection{Babel}

    \dots als populärer Vertreter eines JavaScript-Compilers

\section{Evaluation bestehender Transpilierungs-Ansätze}

  Erläutern, dass es da schon eine Handvoll Ansätze auf GitHub gab, aber die alle nicht einsatzbereit waren.
