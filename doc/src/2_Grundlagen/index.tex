\chapter{Grundlagen}

... und diese werden nun sogleich näher beschrieben:

\section{Statische Typsysteme für JavaScript}

  \subsection{Flow [Basistypen, Hilfstypen, Deklarationen usw.]}
    Flow beschreiben (und zwar mit entsprechender Fachsprache)

    \subsubsection{Basistypen}
    \subsubsection{Hilfstypen}
    \subsubsection{Deklarationen}

  \subsection{TypeScript}
    TS beschreiben (und zwar mit entsprechender Fachsprache)

\section{Quelltext-Transformation durch Transpilierung}

  Was macht eigentlich so ein Transpiler? Hier Theorie (AST etc.)

  \subsection{Theoretische Grundlagen [Parser, AST etc.]}
  \subsection{Babel}

  \dots als populärer Vertreter eines JavaScript-Compilers
