\chapter{Motivation}
\label{chap:motiviation}

\section{Historische Entwicklung}

Als JavaScript 1995 von Brendan Eich innerhalb von lediglich zehn Tagen als Bestandteils des Webbrowsers \enquote{Netscape Communicator} entworfen wurde~\autocite{severance:2012:js10days}, war nicht abzusehen, welch enorme Bedeutung die Sprache über 20 Jahre später inne haben wird: Heute wird JavaScript oft als die am weitesten verbreitete Programmiersprache der Welt betrachtet~\autocite{PAULSON:2007,CROCKFORD:JS_POPULAR}. Dies belegt beispielsweise die alljährliche Umfrage \enquote{\textit{Stack Overflow Developer Survey}} der Programmierer-Plattform \enquote{Stack Overflow}, welche die Ergebnisse der weltweiten Befragung von circa 90.000 Software"=Entwicklern auswertet~\autocite{STACKOVERFLOW:SURVEY}. Bereits das siebte Jahr in Folge führt JavaScript dort die Rangliste der populärsten Programmiersprachen an. Der seit vielen Jahren anhaltende Trend dass zunehmend mehr Software als Webanwendung statt konventioneller Desktop-Anwendung konzipiert wird, hat stark zum Bedeutungszuwachs JavaScripts beigetragen~\autocite{TAIVALSAARI:2017,CASTELEYN:2014}. Neue Konzepte wie \enquote{\textit{Asynchronous JavaScript and XML}} (AJAX)~\autocite{GARRET:AJAX} und \enquote{\textit{Single-page applications}} (SPA) haben Teile der ehemals serverseitigen Anwendungslogik in den Client und damit in die Domäne von JavaScript verlagert. Mit Aufkommen der Laufzeitumgebung \textit{Node.js}~\autocite{SOFTWARE:NODEJS} im Jahr 2009 hat sich die Sprache darüber hinaus von der Lingua Franca des Webbrowsers zu einer Alternative zu etablierten serverseitigen Programmiersprachen wie Java oder C\# entwickelt~\autocite{TILKOV:NODEJS}.

Die heutige große Beliebtheit der Sprache steht in starkem Kontrast zu ihren Anfängen: Aus verschiedenen Gründen wurde JavaScript von professionellen Software"=Entwicklern zu Beginn als eine mangelhaft entworfene Programmiersprache betrachtet~\autocite{CROCKFORD:JS_POPULAR}. JavaScript galt als eine Sprache, die \enquote{nur von Amateuren}~\autocite{CROCKFORD:JS_MISUNDERSTOOD} genutzt wurde und besaß eine zweifelhafte Reputation~\autocite{THIEMANN:2005,THOMAS:2007}. Die uneinheitliche und teilweise fehlerbehaftete Implementierung in den frühen Webbrowsern und der Mangel an Entwicklungswerkzeugen wie Debuggern erschwerte die Programmierung von JavaScript-Anwendung enorm~\autocite{OREILLY:JS_HOW_DID_WE_GET_THERE}. Darüber hinaus behinderte die anfängliche Unbeständigkeit der Sprache und die \enquote{schlechte Qualität}~\autocite{CROCKFORD:JS_MISUNDERSTOOD} der ersten ECMAScript-Spezifikationen~\autocite{ECMASCRIPT:1997} eine breite Akzeptanz der Sprache in der damaligen Entwickler-Szene. \enquote{ECMAScript} ist der Name der formalen Spezifikation von JavaScript und geht auf die Normungsorganisation \textit{Ecma International} zurück. Netscape beauftrage diese im November 1996 mit der Erstellung des Standards, welcher im Jahr darauf unter der Bezeichnung \enquote{ECMA-262} veröffentlicht wurde~\autocite{ECMASCRIPT:1997}.

\section{JavaScripts dynamische Typisierung}

Ein weiterer Aspekt, der oft kontrovers betrachtet wird, ist das dynamische Typsystem von JavaScript, da dieses die Entwicklung sicherer, korrekter Software erschwert und teilweise zu unerwarteten Ergebnissen führt~\autocite{NIKHIL:2014,PRADEL:2015}. Eine dynamische Typisierung ist dadurch charakterisiert, dass jedem Wert der Programmiersprache zur \emph{Laufzeit} ein Typ zugewiesen wird und sich dieser je nach Striktheit des Typsystems zu einem späteren Zeitpunkt implizit oder explizit ändern kann~\autocite[45]{WALDMANN:PPS}. Die \emph{Typsicherheit}, also die Aussage, dass ein Programm keine Regeln des Typsystems verletzt, wird durch dynamische Überprüfungen der Laufzeitumgebung realisiert~\autocite[37]{CARDELLI:TYPE_SYSTEMS}.
In JavaScript wird der Typ von Werten bei Bedarf automatisch, implizit umgewandelt~\autocite{RICHARDS:2010}. Zu Beachten gilt, dass Variablen anders als Werte nicht typisiert sind~\autocite[31]{FLANAGAN:2006}. Einerseits kann eine derartige Flexibilität den Entwicklungsprozess beschleunigen~\autocite[1]{FLOW:PAPER}, andererseits birgt dies das Risiko von Laufzeitfehlern, da Typverletzungen erst bei der Ausführung des Programms erkannt werden können~\autocite{ANDERSON:2005}. Weil nicht immer offensichtlich ist, welche Datentypen in Quelltexten vorliegen, ist die Programmierung fehleranfällig. Problematisch ist auch, dass JavaScript viele Operationen, die traditionell als dynamischer Typfehler gelten, nicht als solche behandelt und die Programmausführung fortsetzt~\autocite{NIKHIL:2014}. Beispielsweise kann auf nicht-existente Attribute von Objekten zugegriffen werden ohne dass dies einen Typfehler verursacht.

% JavaScripts Typsystem wird manchmal auch als \enquote{\emph{schwach}} bezeichnet. Eine solche Kategorisierung in \enquote{schwache} und \enquote{starke} Typsysteme ist umstritten, da keine allgemein anerkannte Definition dieser unpräzisen Klassifizierung existiert~\autocite{CARDELLI:TYPE_SYSTEMS}. Im Allgemeinen wird von starker Typisierung gesprochen, wenn eine sehr strikte Typisierung vorliegt, d.~h. wenn im Extremfall der Typ einer Variablen \emph{lediglich} durch explizite Sprachkonstrukte wie beispielsweise in Haskell umgewandelt werden kann~\autocite{HASKELL}. Werden Typen hingegen wie in JavaScript bei Bedarf implizit in andere Typen konvertiert, so wird dies als schwache Typisierung angesehen.

Die Spezifikation von JavaScript definiert die folgenden sieben Datentypen~\footcite[Abschnitt~6.1]{ECMASCRIPT:2019}. Diese legen fest wie Werte des Programms interpretiert werden und welche Operationen mit ihnen möglich sind. Alle Datentypen außer \enquote{\texttt{Object}} sind dabei primitiv, das heißt nicht weiter zerlegbar (atomar)~\footcite[Abschnitt 4.3.2]{ECMASCRIPT:2019}.

\begin{enumerate}[midpenalty=1000]
  \item \code{Undefined}: Undefinierte Werte
  \item \code{Null}: Nullwert
  \item \code{Boolean}: Boolesche Werte
  \item \code{String}: Zeichenketten
  \item \code{Symbol}: Eindeutige Bezeichner
  \item \code{Number}: Gleitkommazahlen
  \item \code{Object}: Alles Weitere (Felder, Funktionen, Objekte, usw.)
\end{enumerate}

Tabelle~\ref{tab:js-type-coercion} auf der nächsten Seite veranschaulicht einige der überraschenden Effekte der impliziten Typumwandlung von JavaScript, indem verschiedene Ausdrücke mit ihren jeweiligen Gegenstücken in Python~\autocite{PYTHON3} und Ruby~\autocite{RUBY} verglichen werden. Python und Ruby sind dynamisch typisierte Skriptsprachen, die beide wie JavaScript in den Neunzigerjahren entstanden sind (1991 bzw. 1995).
In den ersten drei Zeilen werden Feld- und Objektliterale durch den Konkatenationsoperator \enquote{\texttt{+}} verknüpft. Dieser Operator kann in JavaScript sowohl für die Addition, als auch die Verknüpfung \emph{beliebiger} anderer Typen verwendet werden. Während die Konkatenation von Zeichenketten und die Addition von Zahlen zweckmäßig ist, ist sie für die anderen Datentypen unsinnig, aber durch das dynamische Typsystem JavaScripts möglich. Die Argumente des Operators werden je nach Datentyp und ihrer Reihenfolge entsprechend den Regeln der Spezifikation~\footcite[Abschnitt~12.8.3]{ECMASCRIPT:2019} zuerst implizit umgewandelt und daraufhin durch Addition oder String-Verkettung verknüpft. In Python und Ruby führt die Verwendung des Operators auf diese Weise zu einem Typfehler, da der Ausdruck ungültig ist. Die Ablehnung solcher unsinniger Programme durch das Laufzeitsystem ist nützlich, weil auf diese Weise offensichtliche Fehler unmittelbar erkannt werden können.
In Zeile 4~f. werden jeweils zwei Feldliterale konkateniert. In diesem Fall reihen viele dynamische Programmiersprachen, wie auch Python und Ruby, die Werte des ersten Felds an die des zweiten an. JavaScript hingegen konvertiert die Felder absurderweise zu ihrer Repräsentation als Zeichenkette und konkateniert dieses daraufhin, anstatt die Werte des Feldes zu kombinieren.

% \bigbreak
\begingroup
\setlength{\tabcolsep}{4pt}
\begin{table}[htb]
  \caption[Dynamische Typumwandlung verschiedener JavaScript-Ausdrücke im Vergleich zu Python und Ruby]{Demonstration der dynamischen Typumwandlung verschiedener Ausdrücke in JavaScript im Vergleich zu Python und Ruby (Node 12.6, Python 3.7, Ruby 2.6.).}
  \footnotesize
  \begin{tabu} to \textwidth {@{}p{5.4cm}XX@{}}
    \midrule
    \libertineSB{JavaScript} & \libertineSB{Python} & \libertineSB{Ruby} \\
    \midrule
    \code{\{\} + []}~~$\Rightarrow$~~\code{0} & \code{\{\} + []}~~$\Rightarrow$~~\code{TypeError} & \code{\{\} + []}~~$\Rightarrow$~~\code{TypeError} \\
    \code{[] + \{\}}~~$\Rightarrow$~~\code{'[object Object]'} & \code{[] + \{\}}~~$\Rightarrow$~~\code{TypeError} & \code{[] + \{\}}~~$\Rightarrow$~~\code{TypeError} \\
    \code{\{\} + \{\}}~~$\Rightarrow$\newline\-\hspace*{1.25em}\code{'[object Object][object Object]'} & \code{\{\} + []}~~$\Rightarrow$~~\code{TypeError} & \code{\{\} + []}~~$\Rightarrow$~~\code{TypeError} \\
    \code{[] + []}~~$\Rightarrow$~~\code{''} & \code{[] + []}~~$\Rightarrow$~~\code{[]} & \code{[] + []}~~$\Rightarrow$~~\code{[]} \\
    \code{[1, 3] + [5]}~~$\Rightarrow$~~\code{'1,35'} & \code{[1, 3] + [5]}~~$\Rightarrow$~~\code{[1, 3, 5]} & \code{[1, 3] + [5]}~~$\Rightarrow$~~\code{[1, 3, 5]} \\
    \code{[] == 0}~~$\Rightarrow$~~\code{true} & \code{[] == 0}~~$\Rightarrow$~~\code{false} & \code{[] == 0}~~$\Rightarrow$~~\code{false} \\
    \code{0 == '0'}~~$\Rightarrow$~~\code{true} & \code{0 == '0'}~~$\Rightarrow$~~\code{false} & \code{0 == '0'}~~$\Rightarrow$~~\code{false} \\
    \code{false == 'false'}~~$\Rightarrow$~~\code{false} & \code{false == 'false'}~~$\Rightarrow$~~\code{false} & \code{false == 'false'}~~$\Rightarrow$~~\code{false} \\
    \code{0 === '0'}~~$\Rightarrow$~~\code{false} & \code{0 == '0'}~~$\Rightarrow$~~\code{false} & \code{0 == '0'}~~$\Rightarrow$~~\code{false} \\
\code{typeof null}~~$\Rightarrow$~~\code{'object'} & \code{type(None)}~~$\Rightarrow$~~\code{NoneType} & \code{nil.class}~~$\Rightarrow$~~\code{NilClass} \\
    \midrule
  \end{tabu}
  \label{tab:js-type-coercion}
\end{table}
\endgroup


Problematisch ist weiterhin das Verhalten des Vergleichsoperators \enquote{\texttt{==}}\footnote{Vgl. \textit{Abstract Equality Comparison}~\autocite[Abschnitt 7.2.14]{ECMASCRIPT:2019}.} (Zeile 6 ff.). Auch hier werden die Typen der Operator-Argumente automatisch entsprechend des Algorithmus der Spezifikation umgewandelt, um die Gleichheit der zugehörigen Werte zu überprüfen. Da die Regeln der Typumwandlung kompliziert und unintuitiv sind, verhält sich der Operator in vielen Fällen anders, als es Programmierer erwarten~\autocite{PRADEL:2015}. Deshalb wurde wurde mit der dritten Version der ECMAScript-Spezifikation~\autocite{ECMASCRIPT:1999} 1999 ein strikter Vergleichsoperator \enquote{\texttt{===}} eingeführt, welcher sowohl die Typ- als auch die Wertgleichheit überprüft\footnote{Vgl. \textit{Strict Equality Comparison}~\autocite[Abschnitt 7.2.15]{ECMASCRIPT:2019}.}.
Das letzte Beispiel für die Inkonsistenz des Systems ist schließlich der \enquote{\texttt{typeof}}-Operator, welcher den momentanen Typ einer Variable zurückliefert. Obwohl der Operator den korrekten Wert für alle anderen Fälle zurückliefert, wird für \enquote{\texttt{null}} der mehrdeutige Wert \enquote{\texttt{object}} ausgegeben. Das Resultat der entsprechenden Ausdrücke in Python und Ruby ist eindeutig.

Natürlich stellen die ausgewählten Beispiele keine vollständigen, sinnvollen Programme dar. Dennoch demonstrieren sie einige der fundamentalen Probleme des Typsystems von JavaScript. Die Abwärtskompatibilität der Sprache ist aufgrund ihrer sehr weiten Verbreitung auf vielen verschiedenen Laufzeitumgebungen von immenser Wichtigkeit. Eine nachträgliche, tiefgreifende Anpassung des Sprachkerns und des Typsystems ist äußerst schwierig, da viele bestehende JavaScript"=Programmen hierdurch unbrauchbar würden~\autocite[1]{CROCKFORD:JS_GOOD_PARTS}. Brendan Eich bezeichnete seinen 1995 in kurzer Zeit entwickelten JavaScript-Prototyp 2008 rückblickend als \enquote{fait accompli}~\autocite{EICH:POPULARITY}, also als eine \enquote{vollendete Tatsache}. Viele der anfänglichen Unzulänglichkeiten der Sprache konnten durch Erweiterungen der Spezifikation und zunehmend konsistenter Implementierungen nach und nach behoben werden. Dennoch bleibt die sehr dynamische Ausprägung des Typsystems ein Hindernis für die Entwicklung umfangreicher, sicherer und gut wartbarer JavaScript-Anwendungen. Über die Jahre sind aus diesem Grund verschiedene Ansätze entstanden, um diese Problematik zu überwinden und JavaScript um ein \emph{statisches} Typsystem zu erweitern. Statische Typsysteme zeichnen sich dadurch aus, dass Typfehler bereits \emph{vor} Ausführung eines Programms erkannt werden, indem dessen Quelltexts analysiert wird. Somit kann eine bestimmte Klasse von Bugs frühzeitig erkannt und die Qualität der Software so gesteigert werden. Die Konzepte von Typsystemen und deren statischer Ausprägung werden in Kapitel~\ref{chap:basics} erläutert.

\section{Zielsetzung und Aufbau der Arbeit}

Die vorliegende Masterarbeit beschäftigt sich mit Transpilierung statischer Typsysteme für JavaScript. Dabei werden zwei derzeit populäre Systeme betrachtet, welche eine statische Typisierung in JavaScript ermöglichen: einerseits \textit{Flow}~\autocite{FLOW:PAPER} und andererseits \textit{TypeScript}~\autocite{TYPESCRIPT:SPEC}. Ziel der Arbeit ist die Entwicklung eines Transpilers, der in der Lage ist den gesamten Quelltext eines JavaScript"=Projekts von Flow in äquivalenten TypeScript-Code zu übersetzen. Eine händische Migration umfangreicher Projekte ist impraktikabel, da dies sehr zeitaufwändig und fehleranfällig wäre. Der Wechsel des eingesetzen Typsystems wird innerhalb des Unternehmens \textit{TeamShirts}\footnote{Siehe Abschnitt~\ref{sec:status-quo}.} angestrebt, da angenommen wird, dass TypeScript verschiedener Vorteile gegenüber Flow aufweist. Es wird vermutet, dass TypeScript Typfehler und Bugs in höherem Maße erkennt und externe Software-Bibliotheken und Frameworks besser unterstützt. Weiterhin wird gemutmaßt, dass TypeScript Flow hinsichtlich Stabilität, Performance und Zukunftssicherheit überlegen ist. Die Verifizierung dieser Thesen auf Grundlage empirischer Daten und der gesammelten Erfahrung während der Projektmigration ist Gegenstand der anschließenden Untersuchung.

Der Aufbau der Arbeit gliedert sich wie folgt: Nachfolgend werden zunächst die benötigten Grundlagen hinsichtlich statischer Typsysteme für JavaScript und der Transpilierung von Programmen geschaffen. Anschließend wird die Ausgangslage der JavaScript-Projekte von TeamShirts analysiert, die Ziele des Wechsels von Flow zu TypeScript erläutert und die Anforderungen an den geplanten Transpiler ausgeführt. Daraufhin werden Architektur und Details der Implementierung des Transpilers umfassend betrachtet. In Kapitel~\ref{chap:execution} wird dann die Durchführung der Migration der JavaScript-Projekte mittels des Transpilers beschrieben. Auch werden dort die aufgetretenen Schwierigkeiten und gewonnen Erfahrungen dargelegt. Die Ergebnisse der Migration werden im Anschluss ausgewertet und kritisch diskutiert. Weiterhin wird die Umsetzung vergleichbaren, bestehenden Werkzeugen gegenüber gestellt und hinsichtlich der Zielvorgabe bewertet. Schließlich wird ein Fazit der gesamten Arbeit gezogen und ein Ausblick über mögliche Erweiterungen gegeben.
