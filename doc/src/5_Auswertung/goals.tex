\subsection{Erkennung weiterer Typ- und Programmfehler}
\label{goal:new-type-errors}

Der TypeScript-Compiler besitzt wie in Abschnitt~\ref{sec:typescript} beschrieben verschiedene Optionen, um die Striktheit der Typüberprüfungen zu erhöhen. Diese kann durch Setzen der Option \enquote{\code{strict}} angepasst werden, sodass daraufhin deutlich mehr Ausdrücke als Typfehler betrachtet werden.

% TODO: histogramme mit top 8 fehlern oder so strikt vs nicht-strikt
% TODO: erklärung: wo kommen diese fehler her? welche davon sind tatsächlich problematisch? welche nur bs von ts?
% TODO: ein paar beispiele für echte probleme

\subsection{Unterstützung externer Bibliotheken}

% TODO: vergleich flow-typed definitely typed
% TODO: 3 "schöne" bibs rauspicken (react, redux, lodash, datefns?) und libdefs vergleichen

\subsection{Performance der Typüberprüfungen}

% TODO: messungen (complete) für 4 geräte (tabelle reicht wohl)
% TODO: wenn ts multicore könnte, dann ...

\begin{figure}[tbp]
  \centering

  \input{../data/plots/complete/components-plot}

  \vspace{.5\baselineskip}

  \input{../data/plots/complete/helios-plot}
  \vspace{.5\baselineskip}
  \caption[Einfluss der zur Verfügung stehenden Rechenkerne auf durchschnittliche Laufzeit der Typüberprüfung von Flow und TypeScript]{
    Einfluss der zur Verfügung stehenden Rechenkerne auf durchschnittliche Laufzeit der Typüberprüfung von Flow und TypeScript der Projekte \textit{Components} und \textit{Helios}.
  }

  \vspace{\baselineskip}
  \caption*{
    \small
    Jeweils 100 Messwerte, bestes und schlechtestes Dezil verworfen.\\
    Flow v0.96, TypeScript v3.5.\\
    Intel Core i7-6700 CPU mit 3,4~GHz.
  }
\end{figure}


\subsection{Zukunftssicherheit und Transparenz der Technologie}

% TODO: artikel: what we've been up to
% TODO: roadmap
% TODO: evtl berechnen wie lang issues im schnitt offen bleiben?
