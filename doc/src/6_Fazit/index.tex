\chapter{Schlussbetrachtung}
\label{chap:conclusion}

\section{Zusammenfassung}

In der vorliegende Masterarbeit wurde die Fragestellung behandelt, wie statische Typsysteme für JavaScript automatisch ineinander überführt werden können. Dabei wurden zwei derzeit populäre Systeme betrachtet: Einerseits Flow~\autocite{FLOW:PAPER}, andererseits TypeScript~\autocite{TYPESCRIPT:SPEC}. Zielstellung der Arbeit war es, einen Transpiler zu entwickeln, der beliebige durch Flow typisierte JavaScript-Quelltexte in äquivalenten TypeScript-Code übersetzen kann. Eine derartige Migration von mehreren JavaScript-Projekten wurde innerhalb des Unternehmens Spreadshirt angestrebt, da die Annahme besteht, dass TypeScript verschiedene Vorteile gegenüber Flow aufweist. Zu Beginn der Arbeit wurde evaluiert, ob bereits Software existiert, die eine derartige Übersetzung der Flow-Typisierung nach TypeScript ermöglichen. Tatsächlich bestanden zum damaligen Zeitpunkt zwei Ansätze in einem frühen Entwicklungsstadium, die aber inaktiv schienen und über keinen vollständigen Funktionsumfang verfügten. Da die Lösung der gegebenen Problemstellung somit nicht umsetzbar war, wurde entschieden selbst einen geeigneten Transpiler zu entwickeln.

Die Zielsetzung der Migration zu TypeScript wurden in Zusammenarbeit mit dem Entwicklerteam der Abteilung TeamShirts daraufhin präzisiert: So sollte mit dem Wechsel zum einen neue Typ- und Programmfehler im bestehenden und zukünftigen Code aufgedeckt werden, andererseits die Unterstützung externer Software-Bibliotheken durch das Typsystem verbessert werden. Weiterhin wurde die These aufgestellt, dass TypeScript eine bessere Performance bezüglich der Typüberprüfungen als Flow aufweise und darüber hinaus transparenter entwickelt werde. Neben diesen Zielen wurden außerdem einige technische Anforderungen an den umzusetzenden Flow-Transpiler spezifiziert: Dieser soll in der Lage sein \emph{jegliche} Flow-Typen korrekt in äquivalente TypeScript-Ausdrücke zu übersetzen. Dabei muss die Semantik des Eingabeprogramms beibehalten werden, sodass keine neuen Programmfehler in die Ausgabe eingeschleust werden. Auf Grundlage der Rahmenbedingungen bei TeamShirts wurde weiterhin vorgegeben, dass sowohl aktuelle JavaScript-Syntax gemäß der ECMAScript-Spezikation 2019~\autocite{ECMASCRIPT:2019}, als auch vorläufige Spracherweiterungen wie beispielsweise Klassendekoratoren unterstützt werden müssen. Ferner muss JSX-Syntax eingelesen werden können, da die vorliegenden Projekte von TeamShirts hierauf aufbauen. Zuletzt soll der Transpiler die Verarbeitung gesamter Projektverzeichnisse ermöglichen und eine originalgetreue Formatierung der TypeScript-Ausgabe realisieren.

Anschließend wurde der Flow-Transpiler gemäß diesen Anforderungen auf Basis des Werkzeugs Babel umgesetzt. Babel wurde ausgewählt, da nur hier jede der geforderten Syntaxarten eingelesen und generiert werden kann. Neben der als Babel-Plugin realisierten Programmtransformation wurde auch ein Kommandozeilenprogramm implementiert, das die zentrale Benutzerschnittstelle darstellt und die geforderte Übersetzung gesamter Projekte ermöglicht. Um den Aufwand manueller Korrekturen von Typfehlern nach Ausführung des Transpilers zu reduzieren, wurden in diesen weiterhin einige Optimierungen, wie zum Beispiel die automatische Anpassung von React-Typimporten, integriert. Die umgesetzte Formatierungsroutine auf Grundlage des Werkzeugs Prettier ermöglicht schließlich die größtenteils originalgetreue Formatierung der Ausgabe.

Nachdem der Transpiler fertiggestellt war, wurden die zwei Frontend-Projekte Components und Helios auf diese Weise nach TypeScript übersetzt. Der TypeScript-Compiler stellte daraufhin eine Vielzahl von neuen Typfehlern in beiden Projekten fest. Während der Großteil dieser auf die ausgeführten  grundlegenden Unterschiede von Flow und TypeScript, die Integration von externen Typdeklarationen für Bibliotheken und auf Unzulänglichkeiten der vorherigen Flow-Typisierung zurückgeführt werden können, wurden durch einige der aufgetretenen Typfehler tatsächlich bisher unerkannte Programmfehler aufgedeckt. Wie gezeigt bestehen bei TypeScript im Vergleich zu Flow in der Tat Typdeklarationen für eine insgesamt größere Zahl von Bibliotheken. Jedoch konnten bei Betrachtung einer Auswahl der meistverwendesten Pakete in den zwei Projekten von TeamShirts nur geringfügige Vorteile bei TypeScript festgestellt werden. Anders als zunächst angenommen konnte darüber hinaus nur in einigen Fällen eine bessere Performance der Typüberprüfung von TypeScript gegenüber Flow gemessen werden. Die umfangreiche Parallelisierung der Berechnung durch Flow scheint der Architektur von TypeScript zumeist überlegen. In der Tat wird TypeScript aber wie vermutet transparenter als Flow entwickelt. So existiert hier anders als bei Flow eine öffentlich einsehbare Roadmap, sodass die langfristige Weiterentwicklung von Unternehmen gut eingeschätzt werden kann. Auch gemeldete Programmfehler auf der Plattform GitHub werden bei TypeScript im Allgemeinen schneller behoben.

Mit Hilfe der bestehenden Modultests der zwei Projekte wurde näherungsweise gezeigt, dass der umgesetzte Transpiler die Semantik des Eingabeprogramms nur wo nötig verändert. Wie dargelegt wurde ferner die geforderte Vollständigkeit der Übersetzung der Flow-Typen innerhalb des implementierten Transpilers erreicht. Auch die Äquivalenz der Transformationen konnte größtenteils umgesetzt werden, jedoch treten hier einerseits Fälle auf, die prinzipiell nicht bedeutungsgleich nach TypeScript übersetzt werden können, andererseits konnte für drei Hilfstypen von Flow kein äquivalenter TypeScript-Ausdruck gefunden werden. Beide Aspekte werden anhand umfangreicher Fixture-Tests verifiziert. Zur besseren Einordnung der Ergebnisse wurden die inzwischen lauffähigen vergleichbaren Ansätze von Kikura~\autocite{KIKURA:FLOW_TO_TS} und Barabash~\autocite{BARABASH:FLOW_TO_TS} zur Transpilierung von Flow herangezogen. Dabei konnte nachgewiesen werden, dass diese manche der Testfälle aufgrund von Laufzeitfehlern überhaupt nicht übersetzen können und bei anderen Fällen fehlerhafte, das heißt nicht äquivalente, Transformationen auftreten.
Im Vergleich der vorliegenden Implementierung Kikura und Barabash wurde weiterhin festgestellt, dass beide dieser Ansätze bei bestimmter valider Flow-Syntax mit Laufzeitfehlern abstürzen. Barabash ermöglicht wie der umgesetzte Transpiler die Verarbeitung ganzer Projektverzeichnisse. Allerdings konnten hier schlechtere Ergebnisse hinsichtlich der Formatierung der Ausgabe aufgezeigt werden.

Insgesamt kann die Migration zu TypeScript bei Spreadshirt als erfolgreich betrachtet werden: Wie ausgeführt wurden einerseits die gesetzten Ziele größtenteils erreicht, andererseits die technischen Anforderungen an den Transpiler mehrheitlich erfüllt. Durch den Transpiler konnte der Arbeitsaufwand und die Fehleranfälligkeit des Wechsels des Typsystem deutlich reduziert werden.
Die auf diese Weise migrierten Projekte Components und Helios wurden neu ausgeliefert und werden inzwischen produktiv eingesetzt. Auch das Entwicklerteam von TeamShirts ist laut eigener Aussage sehr zufrieden mit dem vollzogenen Wechsel zu TypeScript.

\section{Ausblick}

Zuletzt soll auf zwei Aspekte dieser Arbeit eingegangen werden, die zukünftig verbessert werden könnten. So weist der umgesetzte Transpiler noch verschiedene Schwächen auf: Insbesondere die originalgetreue Formatierung der Ausgabe sollte optimiert werden, da hier wie gezeigt noch teilweise Fehler auftreten. Das implementierte zeilenbasierte Verfahren funktioniert zwar in den allermeisten Fällen akkurat, ist aber nicht besonders robust. Sobald eine Abweichung beim Vergleich der Ein- und Ausgabe auftritt, ist eine korrekte Behandlung der verbleibenden Datei unmöglich\footnote{Vgl. Abschnitt~\ref{sec:results-formatting}.}. Ein aussichtsreicher Ansatz wäre, die gesamte Quelltext-Transformation durch einen konkreten statt abstrakten Syntaxbaum zu realisieren, sodass alle Information über die Formatierung des ursprünglichen Quelltexts präzise beibehalten wird. Bereits während der Entwicklung des Transpilers wurde dieser Ansatz in Betracht gezogen, jedoch wurde festgestellt, dass lediglich das Projekt \textit{CST} (\textit{Concrete Syntax Tree})~\autocite{SOFTWARE:CST} existiert, welches das Einlesen konkreter Syntaxbäume für JavaScript ermöglicht. Weil hierbei aber weder Flow, noch TypeScript unterstützt werden, konnte dieser Ansatz nicht weiterverfolgt werden. Falls das Projekt entsprechend erweitert werden würde, könnte die Quelltext-Transformation auf diese Weise angestrebt werden.

Aufgrund des begrenzten zeitlichen Rahmens dieser Masterarbeit konnten in den zwei migrierten Projekten von TeamShirts nur die nicht-strikte Überprüfung von Typfehlern durch TypeScript realisiert werden. Zur weiteren Steigerung der Typsicherheit sollte deshalb perspektivisch in beiden Projekten strikte Überprüfungen aktiviert und die resultierenden Typfehler daraufhin korrigiert werden.
