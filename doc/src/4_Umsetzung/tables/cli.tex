\begin{table}[tbh]
  \small
  \begin{tabu} to \textwidth {@{}lX[l]@{}}
    \midrule
    \libertineSB{Option} & \libertineSB{Beschreibung} \\
    \midrule
    \medskip
    \texttt{-V -{}-version} & Versionsnummer anzeigen. \\
    \medskip
    \texttt{-d -{}-dry-run} & Generierten TypeScript-Code auf Standardausgabe statt in Dateien schreiben (Testlauf).\\
    \medskip
    \texttt{-e -{}-exclude-dirs <pattern ...>} & Kommaseparierte Liste von Verzeichnissen, die von der Transpilierung rekursiv ausgeschlossen werden sollen. \\
    \medskip
    \texttt{-i -{}-include-pattern <pattern>} & Wildcard-Muster (\textit{glob pattern}) für Eingabedateien bei Angabe von Verzeichnissen (Standardwert: \texttt{"**/*.{js,jsx}"}). \\
    \medskip
    \texttt{-r -{}-replace} & Originaldateien (Flow) mit generierten TypeScript-Dateien ersetzen, statt diese beizubehalten. \\
    \medskip
    \texttt{-D -{}-replace-decorators} & Klassendekoratoren durch verschachtelte Funktionsaufrufe ersetzen (vgl. Abschnitt~\ref{sec:class-decorators}). \\
    \medskip
    \texttt{-h -{}-help} & Hilfe anzeigen. \\
    \midrule
  \end{tabu}
  \caption{Optionen des Kommandozeilenprogramms (\textit{Reflow}).}
  \label{tab:cli-options}
\end{table}
