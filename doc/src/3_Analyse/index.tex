\chapter{Ziel- und Anforderungsanalyse}
\label{chap:analysis}

Nachdem die Grundlagen der Thematik erörtert wurden, sollen nachfolgend die Ziele der Migration von Flow nach TypeScript innerhalb des Unternehmens \textit{TeamShirts} und die Anforderung an die Implementierung des Transpilers dargelegt werden. Zur Verdeutlichung der Rahmenbedingungen dieser Arbeit wird zunächst die Ausgangslage innerhalb des Unternehmens beschrieben.

\section{Ausgangslage}
\label{sec:status-quo}

% Weiterhin wird der Status Quo zu Beginn der Arbeit hinsichtlich bereits bestehender Ansätze zur Lösung der Problemstellung beleuchtet.

\subsection{Kurzvorstellung der Projekte des Unternehmens TeamShirts}
\label{subsec:ts-projects}

Das Leipziger Unternehmen \textit{sprd.net AG} (\textit{Spreadshirt}) ist ein seit 2002 bestehender Anbieter verschiedener eCommerce-Plattformen, welche die individuelle On-Demand-Bedruckung von Kleidung und Accessoires ermöglichen~\autocite{SPREADSHIRT:ABOUT}. Die Produkte können von den Kunden online durch vorgegebene oder eigene Motive gestaltet und anschließend bestellt werden. Um weitere Zielgruppen wie Sportmannschaften, Vereine, Belegschaften usw. besser anzusprechen, wurde 2014 der Geschäftsbereich \textit{TeamShirts} geschaffen in dessen Kontext diese Arbeit entstanden ist. TeamShirts betreibt verschiedene Webanwendung deren Frontend vorrangig auf JavaScript basiert. Innerhalb des Unternehmens gibt es strategische Überlegungen, die bestehenden Projekte nach TypeScript zu migrieren, um die derzeitige Typisierung mit Flow so zu ersetzen. Die Gründe hierfür werden in Abschnitt~\ref{analysis:goals} ausführlich dargelegt. Der Wechsel zu TypeScript wurde durch den in dieser Arbeit entworfenen und realisierten Transpiler für zwei dieser Projekte umgesetzt. Um die Nachvollziehbarkeit der weiteren Ausführungen zu erleichtern, sollen diese kurz vorgestellt werden.

\subsubsection{Components}

Alle Frontend-Projekte von TeamShirts bauen auf der Software-Bibliothek \textit{React}~\autocite{SOFTWARE:REACT} auf, welche die Programmierung von Benutzeroberflächen auf Basis von sogenannten \emph{Komponenten}\footnote{React-Komponenten sind nicht mit \textit{Webkomponenten}~\autocite{MDN:WEBCOMPONENTS} zu verwechseln. Diese verfolgen ein ähnliches aber nicht identisches Konzept. React interagiert im Gegensatz zu Webkomponenten direkt mit dem \textit{Document Object Model} (DOM) des HTML-Dokuments~\autocite{REACT:WEBCOMPONENTS} statt mit dem sogenannten \textit{Shadow DOM}.} ermöglicht~\autocite{ACM:REACT}. Diese stellen wiederverwendbare, voneinander unabhängige Elemente dar, die aus HTML, CSS und JavaScript aufgebaut sind. Durch Komposition der Komponenten können leicht komplexere Strukturen zusammen gesetzt werden. Das erste Projekt von TeamShirts \textit{Components} stellt eine interne Bibliothek derartiger React-Komponenten dar, die in verschiedenen anderen Software-Modulen eingesetzt wird.

\subsubsection{Helios}

\textit{Helios} ist ein Service für \emph{Server-Side Rendering} (SSR). Unter \emph{Server-Side Rendering} versteht man die Generierung des gesamten HTML-Quelltexts einer Webseite auf Seite des Servers. Dies ist insbesondere für Anwendungen interessant, die JavaScript für den Seitenaufbau verwenden, da sich auf diese Weise verschiedene Performance-Vorteile ergeben. So wird beispielsweise eine schnelle Interaktivität\footnote{engl. \emph{Time to Interactive} (TTI).} der Seite erreicht, da weniger Berechnungen auf Clientseite nötig sind~\autocite{GOOGLE:RENDERING_ON_THE_WEB}. TeamShirts verwendet Helios, um Seiten, die durch verschiedene React-Komponenten aufgebaut sind, serverseitig zu erstellen und auszuliefern.

\subsection{Bestehende Ansätze zur Transpilierung von Flow nach TypeScript}

Vor Beginn der Arbeit wurde evaluiert, ob bereits Lösungsansätze für die gegebene Problemstellung, die Transpilierung der Flow-Typisierung nach TypeScript, bestehen. Tatsächlich existierten zum damaligen Zeitpunkt im Februar 2019 zwei Projekte auf GitHub, welche das gleiche Ziel wie diese Arbeit verfolgen: Bereits im November 2017 begann der Programmierer Boris Cherny mit der Entwicklung eines Babel-Plugins zur Transpilierung von Flow~\autocite{CHERNY:FLOW_TO_TS}. Weiterhin ist im April 2018 ein vergleichbares Projekt durch Yuichiro Kikura entstanden~\autocite{KIKURA:FLOW_TO_TS}. Nach näherer Betrachtung und Erprobung der zwei Plugins wurde jedoch rasch festgestellt, dass sich beide Werkzeuge noch in einem frühen Stadium befinden und deren Funktionsumfang unvollständig ist. Da die Weiterentwicklung der Plugins darüber hinaus nur sporadisch voran getrieben wurde, wurde angenommen, dass die Projekte inaktiv sind. Eine korrekte, ganzheitliche Übersetzung der vorliegenden JavaScript-Quelltexte von TeamShirts war somit nicht umsetzbar. Folglich wurde beschlossen selbst einen Transpiler für Flow, der allen nachfolgend ausgeführten Anforderungen gerecht wird, zu entwickeln.

Einige Monate später im Sommer 2019 ist die Arbeit von Yuichiro Kikura an dessen Plugin wieder aufgenommen worden und es ist ein drittes Projekt der eLearning-Plattform \textit{Khan Academy}~\autocite{KHAN_ACADEMY} hinzu gekommen~\autocite{KHAN:FLOW_TO_TS}. Das etwa zeitgleiche Aufkommen mehrerer Projekte mit der gleichen Zielsetzung wie diese Arbeit kann als Beleg dafür betrachtet werden, dass ein großer Bedarf zu existieren scheint die vorliegende Problemstellung zu lösen. Der in dieser Arbeit umgesetzte Transpiler wird in Abschnitt~\ref{evaluation:other-tools} den konkurrierenden Ansätzen hinsichtlich verschiedener Qualitätskriterien gegenüber gestellt.

\section{Ziele der Migration zu TypeScript}
\label{analysis:goals}

In Zusammenarbeit mit den Software-Entwicklern bei TeamShirts wurden die folgenden Ziele der Migration von Flow nach TypeScript definiert. Diese und die im darauffolgenden Abschnitt dargelegten Anforderungen an den Transpiler bilden die Grundlage für die kritische Bewertung der Ergebnisse in Kapitel~\ref{chap:evaluation}.

\subsection{Erkennung weiterer Typ- und Programmfehler}

Die Hauptmotivation für den Wechsel des Typsystems bei TeamShirts ist die Erkennung weiterer Programmfehler im bestehenden und zukünftigen Code der Frontend-Projekte. Es wird angenommen, dass TypeScript hier, insbesondere auch durch die im nächsten Abschnitt beschriebene Unterstützung externer Bibliotheken, Flow überlegen ist.

Im Umfeld einer eCommerce"=Plattform sind JavaScript"=Laufzeitfehler in der Produktivumgebung fatal, da diese zum Beispiel im vorliegenden Fall Bestellabschlüsse verhindern können und dies in Umsatzeinbußen resultiert. Folglich besteht ein starkes Interesse des Unternehmens die Qualität und Robustheit der betriebenen Software weiter zu steigern, indem bestehende Mängel behoben und zukünftige Programmfehler vermieden werden. Es gilt in der Auswertung zu zeigen, dass TypeScript in der Lage ist weitere Bugs im bestehenden Code durch Typverletzungen aufzudecken.

\subsection{Unterstützung externer Bibliotheken und Frameworks}

Da sämtliche Projekte von TeamShirts auf eine Vielzahl externer Software-Bibliotheken wie React~\autocite{SOFTWARE:REACT} et cetera aufbauen, ist die umfassende Unterstützung dieser durch das Typsystem eine wichtige Zielvorgabe. Sowohl Flow als auch TypeScript ermöglichen es den Funktionsumfang und die Struktur von Bibliotheken und Frameworks durch Typdeklarationen zu beschreiben\footnote{Vgl. Abschnitt~\ref{subsubsec:type-declarations}.}. Damit können auch untypisierte JavaScript-Module nachträglich um eine Typisierung in Flow oder TypeScript erweitert werden, sodass die Vorteile statischer Typsysteme auch hier genutzt werden können. Das Risiko, dass Bibliotheken inkorrekt verwendet werden, kann auf diese Weise erheblich reduziert werden, da beispielsweise fehlerhafte Funktionsaufrufe durch das Typsystem unmittelbar erkannt werden.

Weil es aufwändig und fehleranfällig ist selbst korrekte Typdeklarationen für fremde Bibliotheken anzufertigen, ist es für die tiefgreifende Unterstützung dieser durch das Typsystem entscheidend, dass bereits qualitativ hochwertige Typisierungen extern vorliegen. Qualitativ hochwertig bedeutet, dass die Definitionen vollständig, korrekt und aktuell hinsichtlich der Software-Version sind. Einige Bibliotheken stellen selbst Deklarationsdateien bereit, für andere existieren gemeinschaftlich gepflegte Typisierungen. Es besteht die Vermutung, dass TypeScript im Vergleich zu Flow eine insgesamt größere Zahl von Bibliotheken unterstützt und die zugehörigen Typdeklarationen in mindestens gleichwertiger Qualität vorliegen. Sofern sich diese Hypothese als tatsächlich haltbar erweist, gilt das Ziel als erreicht.

\subsection{Performance der Typüberprüfungen}

Ein weiterer maßgeblicher Aspekt für die Akzeptanz des neuen Systems innerhalb des Entwicklerteams bei TeamShirts ist die Performance, das heißt Schnelligkeit, der Typüberprüfungen. Üblicherweise startet der Editor oder die integrierte Entwicklungsumgebung während des Programmierens einen weiteren Prozess für den Flow- bzw. TypeScript-Sprachserver, welcher das gesamte Projekt und insbesondere offene Dateien im Hintergrund kontinuierlich auf Typverletzungen überprüft. Es ist bedeutsam, dass die Ergebnisse dieser Berechnungen möglichst schnell vorliegen, sodass etwaige Fehler unmittelbar im Editor angezeigt werden können. Falls diese stark verzögert, erst nach einigen Sekunden, ausgegeben werden, so verlangsamt dies den Workflow des Programmierers und beeinträchtigt dessen effizientes Arbeiten. Ein weiteres Erfolgskriterium der Migration zu TypeScript ist damit die Erzielung einer mindestens gleichwertigen Performance.

\subsection{Zukunftssicherheit und Transparenz der Technologie}

Zuletzt sind einerseits die Zukunftssicherheit, andererseits die Transparenz der eingesetzten Technologie von Bedeutung. Hierunter werden innerhalb dieser Arbeit verschiedene Fragestellungen verstanden: Existiert ein öffentlich einsehbarer Projektplan (\textit{Roadmap}) für die Fortentwicklung des Systems? Wie gesichert ist die langfristige Unterstützung des Projekts durch die ursprünglichen Autoren\footnote{\textit{Facebook} (Flow)~\autocite{FLOW:PAPER} bzw. \emph{Microsoft} (TypeScript)~\autocite{TYPESCRIPT:SPEC}.} der Technologie? Werden strategische Entscheidungen öffentlich kommuniziert? Wird die Software quelloffen oder proprietär entwickelt? Werden gemeldete Bugs innerhalb einer angemessenen Zeitspanne behoben? Wird die Gemeinschaft in Entscheidungsprozesse eingebunden?

All diesen Fragen gemein ist das Bedürfnis der Benutzer nach Stabilität, Transparenz und Verlässlichkeit der eingesetzten Technologie. Da die Software-Projekte bei TeamShirts für gewöhnlich über viele Jahre hinweg erweitert und gewartet werden, ist die Beständigkeit und die kontinuierliche Weiterentwicklung des eingesetzten Typsystems von großer Bedeutung für das Unternehmen. Das letzte Ziel der Migration ist deshalb die Verbesserung dieser Aspekte durch den Wechsel zu TypeScript.

\section{Technische Anforderungen an den Transpiler}

Neben den ausgeführten Zielvorgaben wurden weiterhin einige technische Anforderungen an den Transpiler erarbeitet, welche die Grundlage für die spätere Bewertung der Implementierung bilden. Um Sicherzustellen, dass das Werkzeug möglichst generisch einsetzbar ist, wurden die Anforderungen nur wo nötig speziell auf die konkreten Gegebenheiten bei TeamShirts zugeschnitten.

\subsection{Äquivalente und vollständige Übersetzung der Flow-Typen}
\label{subsec:requirement:correct-translation}

Die bedeutsamste Anforderung an den Transpiler ist die äquivalente Übersetzung \emph{sämtlicher} Flow-Typisierung nach TypeScript. Durch die Forderung der Vollständigkeit wird garantiert, dass jedes beliebige Flow-Eingabeprogramm transpiliert werden kann. Da durch die Spezifikation des abstrakten Syntaxbaums von Babel~\autocite{BABEL:PARSER_SPEC} klar definiert ist, welche der Knoten Flow-Syntax darstellen, ist der Umfang einer vollständigen Implementierung exakt eingegrenzt.

Wie in Abschnitt~\ref{sec:static-typesystems-for-js} ausgeführt, werden einige der Funktionen des Typsystems von Flow in TypeScript nicht unterstützt, sodass eine absolut bedeutungsgleiche Übersetzung dieser Typen unmöglich ist. Ein geringer Verlust von Typinformation wird deshalb an dieser Stelle akzeptiert, jedoch soll der Benutzer bei Auftreten dieser Fälle durch eine aussagekräftige Warnung benachrichtigt werden.

\subsection{Semantisch äquivalente Transpilierung des Quelltexts}
\label{subsec:requirement:semantic-equivalence}

Darüber hinaus muss gewährleistet werden, dass die Semantik der ursprünglichen Flow"=Programms nicht durch den Transpilierungsprozess verändert wird, um sicher zu stellen dass keine subtilen, semantischen Fehler in den resultierenden TypeScript-Code eingeführt werden. Das heißt die Gesamtwirkung aller Unterprogramme muss vor und nach der Migration identisch sein.

\subsection{Vollständige Unterstützung moderner JavaScrip- und JSX-Syntax}
\label{subsec:requirement:modern-js-support}

Durch die kontinuierlichen Weiterentwicklung JavaScripts ist die Syntax der Sprache in den letzten Jahren wiederholt um neue Elemente erweitert worden. Beispielsweise wurden mit der achten ECMAScript-Spezifikation 2017 die Schlüsselworte \code{async} und \code{await} eingeführt, welche die asynchrone Programmierung in JavaScript erleichtern~\autocite[430]{ECMASCRIPT:2017}. Damit eine universelle Übersetzung beliebiger Eingaben umsetzbar ist, ist es entscheidend, dass der Transpiler jegliche aktuelle JavaScript-Syntax korrekt verarbeiten kann. Auch die Unterstützung vorläufiger, noch nicht endgültig spezifizierter Spracherweiterungen stellt eine wesentliche Anforderung dar, weil diese im Umfeld einiger populärer Bibliotheken zum Teil bereits heute verwendet werden\footnote{Dies setzt die Benutzung eines Compilers wie Babel voraus, welcher diese experimentelle Syntax vor der Auslieferung in standardkonformes JavaScript übersetzt.}. So wird zum Beispiel die Erweiterung \enquote{\textit{Class field declarations for JavaScript}}~\autocite{ES_PROPOSAL:CLASS_FIELDS} häufig von React-Programmierern benutzt, da deren Syntax die Verwendung mancher Funktionen der Bibliothek vereinfacht~\autocite{REACT:HANDLING_EVENTS}.

Da alle Frontend-Projekte von TeamShirts wie erwähnt auf React basieren, ist die Unterstützung sogenannter JSX-Syntax~\autocite{SOFTWARE:JSX} darüber hinaus eine wesentliche Anforderung an den Transpiler. JSX (\textit{JavaScript XML}) ist eine von React eingeführte syntaktische Erweiterung von JavaScript, die dazu verwendet wird den HTML-Aufbau von Komponenten anzugeben.

\subsection{Verarbeitung gesamter Projektverzeichnisse}
\label{subsec:requirement:batch-processing}

Ein umfangreiche Codebasis besteht üblicherweise aus Hunderten von Einzeldateien. Tabelle~\ref{tab:projects-loc} gibt einen Überblick über die konkrete Anzahl und Zusammensetzung der JavaScript-Dateien im vorliegenden Fall bei TeamShirts. Um alle Dateien eines Projekte sukzessive übersetzen zu können, soll eine Stapelverarbeitung implementiert werden, welche das rekursive Einlesen und Verarbeiten eines gesamten Verzeichnisses realisiert. Dabei soll es auch möglich sein, nur bestimmte Dateien in Unterverzeichnissen ein- bzw. auszuschließen, sodass die Menge der Eingabedateien flexibel eingegrenzt werden kann.

\bigbreak
\begin{table}[tbh]
  \footnotesize
  \begin{tabu} to \textwidth {@{}lrrrr@{}}
    \midrule
    \libertineSB{Projekt} & \libertineSB{Dateien} & \libertineSB{Leerzeilen} & \libertineSB{Kommentarzeilen} & \libertineSB{Codezeilen}  \\
    \midrule
    Components & 331 & 4341 & 963 & 24936 \\
    Helios & 353 & 4814 & 495 & 40127 \\
    \midrule
  \end{tabu}
  \caption{Anzahl von JavaScript-Dateien und Verteilung zugehöriger Leer-, Kommentar- und Codezeilen der zwei Projekte von TeamShirts.}
  \label{tab:projects-loc}
\end{table}


\subsection{Beibehaltung der Quelltext-Formatierung}
\label{subsec:requirement:format}

Als letzte Anforderung wurde schließlich definiert, dass die Quelltext-Formatierung eines Projekte nach Ausführung des Transpilers so originalgetreu wie möglich beibehalten werden muss, da der Programmierstil bedeutsam für die Wartbarkeit und Verständlichkeit von Software ist~\autocite[146]{KERNIGHAN:1982}. Auch bestehen teaminterne Vorgaben bezüglich des Aufbaus und der Formatierung des Codes, welche durch die Migration nicht verworfen werden dürfen. Unter der Formatierung wird nachfolgend die Einrückung des Codes, Zeilenumbrüche und die Position der Leerzeichen und -zeilen verstanden. Auch Block- und Zeilenkommentare müssen korrekt in die TypeScript-Ausgabe übernommen werden.
