\chapter{Ziel- und Anforderungsanalyse}
\label{chap:analysis}

\section{Beschreibung der Ausgangslage}
\label{sec:status-quo}

Was geht so bei TeamShirts? Wie verwenden wir Flow? Wie ist die Typisierung bisher (eher implizit, explizit etc)?

Erläutern, dass es da schon eine Handvoll Ansätze auf GitHub gab, aber die alle nicht einsatzbereit waren.

\section{Ziele der angestrebten Migration zu TypeScript}

Hypothesen, Wünsche, Hoffnungen

  \subsection{Erkennung neuer Bugs und Typfehler}
  \subsection{Unterstützung externer Bibliotheken und Frameworks}
  \subsection{Stabilität und Geschwindigkeit des Typsystems}
  \subsection{Zukunftssicherheit und Transparenz der Technologie}

\section{Anforderungen an den Transpiler}

  \subsection{Korrekte Übersetzung der Flow-Typen nach TypeScript}
  \label{subsection:requirement:correct-translation}

  \subsection{Semantisch äquivalente Transpilierung des Quelltexts}
  \label{subsection:requirement:semantic-equivalence}

  \subsection{Vollständige Unterstützung moderner JavaScript- und JSX-Syntax}
  \label{subsection:requirement:modern-js-support}

  JSX~\footnote{Siehe~\citetitle{SOFTWARE:JSX}~\autocite{SOFTWARE:JSX}.}

  \subsection{Verarbeitung gesamter Projektverzeichnisse}
  \label{subsection:requirement:batch-processing}

  \subsection{Beibehaltung der Quelltext-Formatierung}
  \label{subsection:requirement:format}
