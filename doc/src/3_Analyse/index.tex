\chapter{Ziel- und Anforderungsanalyse}
\label{chap:analysis}

\section{Ausgangslage}
\label{sec:status-quo}

% Bevor die Ziele der Migration nach TypeScript erläutert und die Anforderungen an den Transpiler spezifiziert werden, soll zunächst die Situation vor Beginn der Entwicklungsphase beschrieben werden.

\subsection{Kurzvorstellung der Projekte von TeamShirts}

Das Leipziger Unternehmen sprd.net AG (\textit{Spreadshirt}) ist ein seit 2002 bestehender Anbieter verschiedener eCommerce-Plattformen, welche die individuelle On-Demand-Bedruckung von Kleidung und Accessoires ermöglichen~\autocite{SPREADSHIRT:ABOUT}. Die Produkte können von den Kunden online durch vorgegebene oder eigene Motive gestaltet und anschließend bestellt werden. Um weitere Zielgruppen wie Sportmannschaften, Vereine, Belegschaften et cetera besser anzusprechen, wurde 2014 der Geschäftsbereich \textit{TeamShirts} geschaffen.
TeamShirts betreibt verschiedene Webanwendung deren Frontend mit JavaScript und Flow umgesetzt wurde. Innerhalb des Unternehmens gibt es strategische Überlegungen, die bestehenden Projekte nach TypeScript zu migrieren, um die derzeitige Typisierung durch Flow mit TypeScript zu ersetzen. Die Gründe hierfür werden in Abschnitt~\ref{analysis:goals} ausführlich dargelegt. Der Wechsel zu TypeScript wurde durch den in dieser Arbeit entworfenen und realisierten Transpiler für zwei dieser Projekte umgesetzt. Um die Nachvollziehbarkeit der weiteren Ausführungen zu erleichtern, sollen diese kurz vorgestellt werden.

\subsubsection{Components}

Alle Frontend-Projekte von TeamShirts basieren auf der Bibliothek \textit{React}~\autocite{SOFTWARE:REACT}, welche die Programmierung von Benutzeroberflächen auf Basis von sogenannten \emph{Komponenten}\footnote{React-Komponenten sind nicht mit \textit{Webkomponenten}~\autocite{MDN:WEBCOMPONENTS} zu verwechseln. Diese verfolgen ein ähnliches aber nicht identisches Konzept. React interagiert im Gegensatz zu Webkomponenten direkt mit dem \textit{Document Object Model} (DOM) des HTML-Dokuments~\autocite{REACT:WEBCOMPONENTS}.} ermöglicht~\autocite{ACM:REACT}. Diese stellen wiederverwendbare Elemente dar, die aus HTML, CSS und JavaScript aufgebaut sind sind. Da Komponenten voneinander unabhängig sind können sie leicht zu komplexere Strukturen zusammen gesetzt werden. Das erste Projekt von TeamShirts \enquote{Components} stellt eine Bibliothek von derartigen React-Komponenten dar, die in verschiedenen anderen Projekten eingesetzt wird.

\subsubsection{Helios}




\subsection{Bestehende Ansätze zur Transpilierung von Flow nach TypeScript}




% Was geht so bei TeamShirts? Wie verwenden wir Flow? Wie ist die Typisierung bisher (eher implizit, explizit etc)?

% Erläutern, dass es da schon eine Handvoll Ansätze auf GitHub gab, aber die alle nicht einsatzbereit waren.

\section{Ziele der angestrebten Migration zu TypeScript}
\label{analysis:goals}

Hypothesen, Wünsche, Hoffnungen

  \subsection{Erkennung neuer Typ- und Programmfehler}
  \subsection{Unterstützung externer Bibliotheken und Frameworks}
  \subsection{Stabilität und Geschwindigkeit der Typüberprüfungen}
  \subsection{Zukunftssicherheit und Transparenz der Technologie}

\section{Anforderungen an den Transpiler}

  \subsection{Korrekte Übersetzung der Flow-Typen}
  \label{subsection:requirement:correct-translation}

  \subsection{Semantisch äquivalente Transpilierung des Quelltexts}
  \label{subsection:requirement:semantic-equivalence}

  \subsection{Vollständige Unterstützung moderner JavaScript- und JSX-Syntax}
  \label{subsection:requirement:modern-js-support}

  JSX\footnote{Siehe~\citetitle{SOFTWARE:JSX}~\autocite{SOFTWARE:JSX}.}

  \subsection{Verarbeitung gesamter Projektverzeichnisse}
  \label{subsection:requirement:batch-processing}

  \subsection{Beibehaltung der Quelltext-Formatierung}
  \label{subsection:requirement:format}
